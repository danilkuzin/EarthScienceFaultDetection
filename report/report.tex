\documentclass[11pt,a4paper]{article}
\usepackage[backend=biber]{biblatex}

\addbibresource{references.bib}
\begin{document}
\title{Fault Detection}
\author{Danil Kuzin}
\date{January-March 2019}
\maketitle

\abstract
This report presents the results of the work done for geological faults detection with deep learning methods.



\section{Introduction}

The relevant literature includes: training CNNs on synthetic and real seismic data for faults detection \cite{pochet2018seismic, araya2017automated, xiong2018seismic, chehrazi2013seismic, lu2018using}

The other literature on detection from satellite images includes: detecting roads on pixel level from lots of satellite images and then denoising to get connected nets \cite{mnih2010learning},

The potential extensions of this work can include synthetic data, that is described in \cite{hale2014}.

\section{Data}

We use the satellite images from the Landsat-8 and elevation data from the Opentopography:  Shuttle Radar Topography Mission (SRTM GL1) Global 30m. The additional band is the slope estimated based on elevation.
The faults and fault lookalikes were labelled on some of the images.

\section{Neural network}
We use the LeNet-5 architecture for the problem.

\section{Results}

\section{NN visualisations}

\printbibliography

\end{document}