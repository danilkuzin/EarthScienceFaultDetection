\documentclass[11pt,a4paper]{article}
\usepackage[backend=biber]{biblatex}

\addbibresource{references.bib}
\begin{document}
\title{Fault Detection}
\author{Danil Kuzin}
\date{January-March 2019}
\maketitle

\abstract
This report presents the results of the work done for geological faults detection with deep learning methods.



\section{Introduction}

The relevant literature includes: training CNNs on synthetic and real seismic data for faults detection \cite{pochet2018seismic, araya2017automated, xiong2018seismic, chehrazi2013seismic, lu2018using}

The other literature on detection from satellite images includes: detecting roads on pixel level from lots of satellite images and then denoising to get connected nets \cite{mnih2010learning},

The potential extensions of this work can include synthetic data, that is described in \cite{hale2014}

%* Works on lines detection:
%  * [1](http://openaccess.thecvf.com/content_cvpr_2018/papers/LaLonde_ClusterNet_Detecting_Small_CVPR_2018_paper.pdf)
%  * [2](https://www.cv-foundation.org/openaccess/content_cvpr_2015/papers/Zhang_Cross-Scene_Crowd_Counting_2015_CVPR_paper.pdf)
%  * [3](https://www.cv-foundation.org/openaccess/content_cvpr_2016/papers/Zhang_Single-Image_Crowd_Counting_CVPR_2016_paper.pdf)
%  * [4](http://openaccess.thecvf.com/content_cvpr_2018/papers/Liu_DecideNet_Counting_Varying_CVPR_2018_paper.pdf)

\printbibliography

\end{document}